
{\label{0Kapitel 1: Irgendwo im Nirgendwo}\vspace{0.5cm}\noindent\LARGE Kapitel 1: Irgendwo im Nirgendwo}
\renewcommand{\storychapter}{Kapitel 1: Irgendwo im Nirgendwo}

\vspace{0.5cm}\noindent
Seit drei Wochen sind wir nun schon auf hoher See ohne auch nur eine Landlinie am Horizont. Das Wasser ist inzwischen brackig und ich trinke es nur noch mit verschlossenen Augen, um die kleinen Würmchen darin besser ignorieren zu können. Zum Glück habe ich noch Rum, um den Geschmack zu vertreiben. Unser Essen ist zwar auch bei weitem noch nicht knapp, doch es schmeckt von Tag zu Tag schlechter und abwechslungsreich ist es schon lange nicht mehr.

\vspace{0.5cm}\noindent
Vier Tage Flaute, doch obwohl wir kaum voran kommen, ist mir das lieber, als das Biest von einem Wetter, das wir davor hatten. Und so ist wenigstens Zeit, das beschädigte Royalsegel zu flicken.

Ich versuche, die Dinge positiv zu sehen -- sonst hätte ich mir längst die Kugel gegeben\footnote{Fußnote}:

Immerhin laufen wir hier draußen wenig Gefahr, von hoheitlichen Segeln überrascht zu werden. Der Preis der Freiheit ist wohl, dass andere uns dafür am Galgen baumeln sehen wollen. Doch wo sonst könnte ich -- geboren ohne das erforderliche Zubehör in der Unterbekleidung -- meiner Leidenschaft nachgehen. Und wo sonst könnte ich das unter weiblicher Führung tun, daher segle ich mit Stolz unter dem Jolly Roger.

Von klein auf faszinierten mich Schiffe. Geboren in Gibraltar als Tochter eines Marineoffiziers, schlich ich mich so oft es ging hinaus, um die Flotte der Kriegsmarine zu bestaunen und von der unendlichen Weite des Ozeans zu träumen -- viel mehr gab es dort ohnehin nicht zu tun. Irgendwann ging es dann gänzlich mit mir durch und ich verdingte mich als Schiffsjunge auf einer stattlichen Brigg. Damals ging das noch, doch ich kann von Glück sagen, dass ich, bevor es zu einem ernsthaften Problem wurde, unter die Fittiche{\todo{Was sind Fittiche?}} Kapitänin Maheras gelangte -- auch wenn die Umstände gewaltsamer Natur waren. 

Mahera hat eine eingeschworene Besatzung, die ihr treu folgt. Nur gelegentlich sind neue Leute an Bord und die meisten nur für eine oder wenige Fahrten. Ich bin zwar nun schon fast ein Jahr unter ihrem Kommando, doch immer noch nicht offiziell aufgenommen worden. Ich hoffe sehr darauf, dies zu erreichen\wiggle{ -- je eher, desto besser}{\todo{Soll dieser Teil gestrichen werden? Und wie mache ich diesen Kommentar noch unnötig viel länger? Es hätte ja auch \enquote{Streichen} getan.}}.

\vspace{0.5cm}\noindent

{\label{1Kapitel 2: Das Ziel der Reise}\vspace{0.5cm}\noindent\LARGE Kapitel 2: Das Ziel der Reise}
\renewcommand{\storychapter}{Kapitel 2: Das Ziel der Reise}

\vspace{0.5cm}\noindent
Doch genug von mir\footnote{Noch eine Fußnote}. Ich möchte von einer Entdeckung berichten, die ich gestern gemacht habe und die verspricht, dass die Dinge bald wieder sehr aufregend werden. Nach dem Morgenappell hatte meine Kapitänin mich zu sich bestellt. Was sie von mir wollte, ist mir dabei nicht klar geworden, doch ich konnte nicht umhin die Papiere und Karten auf ihrem mächtigen Schreibtisch zu bemerken -- ein Segen (oder Fluch?) des Lesens mächtig zu sein. Besonders ein mitgenommen aussehendes Schriftstück bannte meine Aufmerksamkeit. War das eine Wegbeschreibung? Reime? Als Mahera meines Blickes gewahr wurde, verdeckte sie rasch das Schriftstück und eine Seekarte, auf der ein Kurs verzeichnet war. Mit einem Mal war sie nur noch kurz angebunden und entließ mich bald wieder. Dieses Verhalten deckte sich mit ihrer Geheimniskrämerei der letzten Zeit und stand ganz im Kontrast zu ihrer sonst so offenen Art, bei der sie häufig die Besatzung in Entscheidungen einband.

Meine Neugier war geweckt. Ich versuchte mich den Tag über mit besonderem Fleiß in meine Aufgaben zu stürzen, doch gelang es mir kaum, mich abzulenken, zumal es gerade ohnehin nicht viel zu tun gab. Mir war bewusst, was für einen Vertrauensbruch ich da ausgeheckt hatte und welchem Risiko ich mich damit aussetzte, doch ich konnte einfach nicht anders. Ich wartete noch bis zum Einbruch der Dunkelheit, um die Gefahr der Entdeckung zu verringern, dann begab ich mich auf das Achterdeck (hier bei uns herrschen nicht so strenge Regeln wie auf den Schiffen der Kriegsmarine). Ich versuchte, so beiläufig und unauffällig wie möglich dabei auszusehen, doch hatte ich das Gefühl mein Vorhaben stünde mir in riesigen Lettern auf die Stirn gedruckt (mal angenommen, die Meisten hier könnten lesen). Mein Herz pochte und meine Hände hinterließen schwitzige Abdrücke an der Reling. In einer missglückten Imitation völliger Gelassenheit lehnte ich mich im hinteren Teil des Achterdecks angekommen an eben diese. Von hier konnte ich das derzeit sehr gemächliche Treiben der anderen beobachten und versuchen, neuen Mut zu schöpfen. Gerade hatte ich mich etwas gefangen, da schreckte mich eine Stimme aus meinen Gedanken hoch. Ich hatte den alten McNolte -- niemand wusste seinen Vornamen, vermutlich nicht einmal er selbst -- nicht kommen sehen. \enquote{Alles in Ordnung mit dir? Siehst etwas mitgenommen aus.}

\enquote{Hab nur ein bisschen viel getrunken und brauche 'nen Moment.}, murmelte ich. Da lachte der Alte und stakte auf seinem Holzbein davon. \enquote{Wärst lieber beim Rum geblieben, das Wasser ist echt die Seuche!}

Beinahe hätte ich auch gelacht.

Jedenfalls war die Luft nun rein und ich nutzte meine Gelegenheit. Da sonst die Takelage mein bevorzugter Aufenthaltsort war, bereitete es mir keine größeren Schwierigkeiten an der griffigen verzierten Bordwand hinabzuklettern und einen Blick durch die Fenster der Kapitänskajüte zu erhaschen.

Ich sah Kapitänin Mahera an ihrem Schreibtisch über das kuriose Schriftstück gebeugt. Auch die Karte war wieder aufgedeckt. Das Schriftstück würde ich so nicht lesen können. Auf der Karte erkannte ich unseren aktuellen Kurs. Doch war in Rot eine Abweichung verzeichnet. Die rote Linie machte einen Schlenker in Richtung der westafrikanischen Küste. Die Kursänderung konnte nicht weit von unserer derzeitigen Position sein. Vielleicht waren wir sogar schon auf dem abweichenden Kurs. Auf der alternativen Route war eine Stelle mit einem {\boldfont\textbf{X}} versehen. Mitten im Meer? Was sollte da so besonders sein? Daneben stand etwas geschrieben. Ich musste mich stark konzentrieren, um Maheras geschwungene Handschrift entziffern zu können, doch dann hatte ich es: Totenkopfinsel. Sollte sich dort eine nicht kartierte Insel befinden? Dann konnte sie wahrlich nicht groß sein. Doch was wollte Mahera dort? Wieso hatte sie uns bisher nichts davon verraten? Und warum zur Hölle trug die Insel diesen Namen?

Es half alles nichts, ich musste erfahren, was auf dem geheimnisvollen Schriftstück stand. Vielleicht würde es mir in der Nacht gelingen, wenn Mahera schlief. Zunächst kletterte ich wieder an Deck. Es fiel mir schwer, abzuwarten und der restliche Abend wurde zur Qual. Obwohl ich noch nicht gegessen hatte, brachte ich keinen Happen herunter. Ich war auch nicht in der Stimmung, mich mit den anderen zu unterhalten, daher hielt ich mich abseits. Das Warten wurde dadurch nur noch zäher.

Irgendwann war es endlich so spät, dass ich beschloss, ein weiteres Mal mein Glück zu versuchen. Und es blieb mir treu. Mahera schien tief und fest zu schlafen. Der Wärme der hiesigen Nächte zum Dank hatte sie die Läden nicht verschlossen. Vorsichtig ließ ich mich herab und schlich zum Schreibtisch. Das fragliche Dokument lag noch an Ort und Stelle, sodass ich es gleich fand, doch war es im Dunkel der Kajüte schwierig zu entziffern. Ich hielt es vor das Fenster ins Mondlicht und kämpfte mich Wort für Wort voran. Ich hatte das Gefühl, dass es eine halbe Ewigkeit dauerte, doch Mahera blieb ruhig und ich wollte nicht so kurz vor dem Ziel aufgeben.

Endlich war es geschafft und meine erste Vermutung bewahrheitete sich:

\vspace{0.5cm}\noindent
\begin{itshape}Fahre ein ins Maul des Todes sacht

Empor an Messers Schneide

\vspace{0.5cm}\noindent
Am toten Baum der Affen Acht

Wo ist die Hand aus Kreide

\vspace{0.5cm}\noindent
Nimm Rat bei dem der Ehrfurcht hat

Find glänzendes Geschmeide\end{itshape}

\vspace{0.5cm}\noindent
Ein Schatz! Das war es also, hinter dem Mahera her war. Und vermutlich wollte sie die Beute ganz für sich behalten, weswegen niemand an Bord eingeweiht war -- oder hatte sie sich für einen Teil der Beute Unterstützung und Stillschweigen erkauft?

Fibrig brannte die Aufregung in mir, gepaart mit der Enttäuschung über unsere wider Erwarten selbstsüchtige Kapitänin. Beinahe hätte ich laut aufgeschrien.

Mühsam beherrscht legte ich das Schriftstück zurück und verließ den Raum durch das Fenster.

Was sollte ich tun? Konnte ich mich der übrigen Besatzung anvertrauen? Was würden sie über mich denken? Ein falsches Wort zur falschen Zeit konnte schnell mein Ende bedeuten. Also entscheide ich mich für Schweigen.

\vspace{0.5cm}\noindent

{\label{2Kapitel 3: Landung}\vspace{0.5cm}\noindent\LARGE Kapitel 3: Landung}
\renewcommand{\storychapter}{Kapitel 3: Landung}

\vspace{0.5cm}\noindent
So kam es, dass wir nur zwei Tage später Land sahen. Ein kleines Fleckchen Erde, das tatsächlich von der Form an das Symbol erinnerte unter dem wir segelten. Mahera sagte nur knapp, dass sie dort etwas zu erledigen hätte und gab Anweisung südlich der Insel zu ankern. Mir war gleich klar, warum sie diesen Ankerpunkt wählte. Zu meiner Verwunderung stellte niemand Fragen, was das sollte. Dies war die denkbar schlechteste Möglichkeit zur Landung -- zwischen scharfkantigen Felszähnen hindurch in Erwartung einer steilen Küste.

Als wir geankert hatten, gab sie Anweisung ein Beiboot zu Wasser zu lassen und machte sich tatsächlich allein auf zwischen den scharfkantigen Felsen hindurchzurudern.

Mühsam beherrscht ballte ich die Hände neben dem Körper zu Fäusten und sah ihr zu, wie sie aus dem Sichtfeld entschwand. Schließlich konnte ich nicht mehr anders und rief mit lauter, leicht vor Zorn bebender Stimme aus:

\enquote{Hört mal her, Leute! Ich weiß, ihr alle liebt eure Kapitänin. Doch was würdet ihr sagen, wenn sie eure Liebe mit Füßen tritt?}

Ich war zwar noch kein fester Teil der Besatzung, doch nun schon immerhin ein Jahr mit diesem bunten Haufen unterwegs. Ich hatte viele Freundschaften geschlossen und fast alle hatten auf irgendeine Weise ihren Platz in meinem Herzen gefunden. Dies war mein Zuhause, dies war meine Familie. Und nun hatte ich ihre Aufmerksamkeit.

\enquote{Heimlich schleicht sie sich davon, um uns unsere wohlverdiente Beute zu verwehren, getrieben von Gier. Wir erdulden nun seit Wochen schon Widrigkeiten, damit sie sich die Taschen voll machen kann. 

\noindent
Doch ich weiß, wo der Schatz versteckt liegt. Ich sage, stellen wir sie zur Rede! Ich sage, holen wir uns unseren Teil der Beute!}

Ich wusste nicht, welche Reaktion ich von den anderen erwarten konnte, doch damit hatte ich nicht gerechnet:

Einstimmiger Jubel aus zahlreichen Kehlen. Mein Kopf brannte feuerheiß -- nicht nur von Zorneshitze.

\enquote{Lasst die Beiboote ins Wasser!}

Eilig wurde etwas Ausrüstung und Proviant zusammengepackt und die Boote herabgelassen. Fast die gesamte Besatzung war nun auf dem Weg durch das Felsenlabyrinth. Eine Minimalbesatzung war an Bord verblieben, um dort handlungsfähig zu bleiben. Bald hatten wir die Steilküste erreicht und aus der Nähe konnte ich Stufen entlang der Felswand entdecken. Dies musste es sein, empor an Messers Schneide. Langsam, einer nach der anderen, arbeiteten wir uns nach oben und sammelten uns auf einem Platteau. Ich war als erstes gegangen und während ich verschnaufte, ließ ich einen Augenblick die überwältigende Aussicht auf mich wirken. So hoch über dem Meer. Unser Schiff, die Mary, wirkte wie ein Spielzeug von hier oben. Ich riss mich von dem Anblick los und wandte mein Auge ins Landesinnere, um den dritten Hinweis zu entschlüsseln. 

Konnte es so einfach sein? Wenige hundert Meter von uns streckte ein gewaltiger Baum seine kahlen Äste in den Himmel. Von Affen war keine Spur, doch dieser tote Gigant schien das einzige auf die Beschreibung passende Landschaftsmerkmal. Daher setzte ich die anderen in Kenntnis, dass unser Weg dort hin führen würde. Wir tauchten ein in eine fremde grüne Welt. Durch das dichte, schattige Blattwerk war es schwer, die Richtung beizubehalten -- einen Kompass hatten wir, entgegen unserer Kapitänin, leider nicht. Dennoch gelang es uns ohne großes Irren, den kahlen Riesen zu finden. Zu seinem Fuß erstreckte sich ein kleiner See. Im und am Wasser tobten unzählige kleiner Affen unter wildem Geschrei umher. Seltsamerweise interessierten sie sich kaum für uns. Ich informierte die anderen, das wir nach einer Hand aus Kreide Ausschau hielten, und machte mich selbst am nackten Stamm des toten Baumes auf die Suche. Nachdem ich dort nichts entdeckte, machte ich mich daran, die Gesteinsblöcke zu untersuchen, die hier verstreut lagen, und wurde schließlich fündig. Auf einem der Blöcke war eine weiße Hand abgebildet, die mit ausgestrecktem Zeigefinger in Richtung der Berge zeigte. 

Auf einem schmalen, kaum sichtbaren Pfad bewegten wir uns immer weiter hinauf, begleitet von der Unsicherheit, ob wir noch auf dem richtigen Weg seien. Es mussten nun schon etwa zwei Stunden vergangen sein, als sich, wie eine klaffende Wunde in der fremdartigen Natur, eine riesige Schlucht vor uns auf tat und uns den Weg abschnitt. Wir beschlossen, eine Pause einzulegen, um uns von dem schweißtreibenden Aufstieg zu erholen und über das weitere Vorgehen zu beraten.

In welche Richtung sollten wir von hier gehen? Die Schlucht zu überqueren schien unmöglich, doch der Fingerzeig musste eindeutig in diese Richtung weisen und wir hatten keinerlei Anhaltspunkt über eine Richtungsänderung. Der nächste Hinweis lautete \enquote{Nimm Rat bei dem der Ehrfurcht hat}, doch was sollte das bedeuten. Die anderen Hinweise waren stets so deutlich gewesen und jetzt wussten wir nicht einmal, ob es der richtige Zeitpunkt war, den Hinweis anzuwenden, was auch immer er besagen sollte. Die anderen wirkten bald so, als ob sie gar nicht mehr über das Problem nachdachten. Sie saßen einfach nur da und starrten ins Leere oder unterhielten sich leise.

Ich zerbrach mir den Kopf über die Botschaft und darüber, ob wir tatsächlich den richtigen Weg genommen hatten, doch nichts brachte mich weiter. Einem mir noch immer nicht begreiflichen Impuls folgend, stand ich schließlich auf und ging zum Rand der Schlucht. Dann starrte ich hinab in die Tiefe. Ehrfürchtig? Ich weiß es nicht. Doch da entdeckte ich etwas. Unter mir an der Felswand waren weiße Stellen zu sehen und bald war ich mir sicher, dass es sich um Schriftzeichen handelte. Von hier oben ließ sich jedoch nicht ausmachen, was sie besagten.

Daher hatte ich wenige Augenblicke später ein Seil um mich geschlungen, das einige der Anderen festhielten, und machte mich an den Abstieg. Trotz des muligen Gefühls aufgrund der bodenlosen Tiefe unter mir, erreichte ich die verborgene Nachricht. Sie zeigte zwei Reihen mit je zwei Symbolen. In der ersten war ein Pfeil nach oben und daneben eine zweistämmige Palme abgebildet. Darunter ein Pfeil nach links und schließlich ein X.

\vspace{0.5cm}\noindent
\begin{figure}[!ht]
\centering
\includegraphics[max height=1.0\textheight,max width=1.0\textwidth]{img/secret-message.png}
\end{figure}

Mir schien dies recht eindeutig. Nachdem ich also oben war, gingen wir wieder zurück auf dem Weg, den wir gekommen waren, nur achteten wir diesmal mehr auf die Vegetation. Nach etwa dreihundert Schritten entdeckten wir eine zweistämmige Palme. Hier musste es sein. Wir schlugen uns nach links in das Unterholz. Ich ermahnte die anderen zur Vorsicht und so schlichen wir -- so gut das mit einer derart großen Gruppe eben möglich ist -- voran, bis sich ganz unerwartet vor uns das Dickicht lichtete. Nun wurde auch klar, woher das Rauschen kam, das wir seit einer Weile gehört hatten. Vor uns lag ein See und dahinter ragte eine große Felswand auf, von der ein Wasserfall hinunter schoss. In der Felswand war zudem eine große Einmündung zu sehen. Da sie nicht vom Ufer aus erreichbar war, wateten wir durch das nur etwa hüfthohe Wasser und erklommen am anderen Ende die glitschigen Felsen, während der Sprühnebel des Wasserfalls uns vollends durchweichte. Ich ging stets voran und bedeutete den anderen mit einem Handzeichen Abstand zu halten. Mit geladener Pistole schlich ich voran.

Ein etwa menschenbreiter, natürlicher Gang führte in das Innere des Gesteins. Das von außen hereindringende Licht wurde immer spärlicher, doch als es kaum noch ausreichte, etwas zu erkennen, sah ich etwas weiter vorne einen rötlichen Schimmer hinter einer Biegung hervordringen.

Noch vorsichtiger als ohnehin schon tapste ich um die Biegung und sah in einer von Fackeln erleuchteten großen Grotte schließlich Mahera lässig auf einem Stapel Kisten sitzen mit einem großen Krug in ihrer Hand. Sie hatte mich wohl gleich entdeckt, denn mit lauter Stimme rief sie:

\enquote{Wärst du alleine gekommen, hätte ich dich erschießen müssen. Aber ich habe die Bande schon gehört. Ja, ganz recht: kommt alle herein!}

Ich war verwirrt, doch meine Waffe hatte ich immer noch fest auf Mahera gerichtet. Da kam der alte McNolte hinter mir ins Licht getreten und sagte mit einem Wink in Richtung der Pistole: \enquote{Leg mal das Ding beiseite, damit ich dir ordentlich gratulieren kann. Du bist jetzt eine von uns.}

Unsicher ließ ich die Pistole etwas sinken und blickte von einer zum anderen, doch da kamen auch schon die anderen unter lauten Jubelrufen und hoben mich, ehe ich mich versah, hoch über ihre Köpfe. Ich wusste gar nicht so recht, wie mir geschah, doch floss bald reichlich Rum. Ein Festmahl mit frischem Obst und gebratenem Erdferkel wurde improvisiert, sodass ich mich vom Trubel treiben ließ und zunächst gar nicht mehr versuchte, die Lage zu erfassen. Am meisten Freude bereitete mir das köstliche, frische Wasser ohne Würmer.

Ich erfuhr im Laufe des Abends, dass die anderen die gesamte Zeit über eingeweiht gewesen waren und die ganze Scharade mitgespielt hatten, um mich auf die Probe zu stellen.

\enquote{Und bevor du enttäuscht bist oder mich der Lüge bezichtigst}, sagte Mahera, nachdem sie mir ein weiteres Mal zugeprostet hatte, \enquote{hier nun das Beste an der Sache: den Schatz gibt es wirklich und du hast dir nun deinen Teil daran verdient. Aber bevor du gleich die Hände aufhältst wisse, dass wir uns darauf geeinigt haben, die Beute gemeinsam zu verwalten. So sichern wir uns auch in schwierigen Zeiten unser Überleben. Und ja, ich weiß, ich trage diesen furchteinflößenden Hut, doch ist das überwiegend zur Schau und ich betrachte mich daher nicht als wichtiger, als jeder und jede andere von uns. Also hör gefälligst auf, dich so ehrfürchtig zu gebähren. Das ist ein Befehl!}

Dies war wahrlich der glücklichste Tag in meinem gesamten bisherigen Leben. Und ich bin sicher, dass noch viele weitere Abenteuer auf mich warten.